% O tipo de documento e as opções de formatação da página
\documentclass[twocolumn,A4]{article}

% Pacotes que vamos utilizar (esses estão em quase todos os documentos)
% %%%%%%%%%%%%%%%%%%%%%%%%%%%%%%%%%%%%%%%%%%%%%%%%%%%%%%%%%%%%%%%%%%%%%%%%%%%%%
% Para poder usar caracteres especiais e acentos
\usepackage[utf8]{inputenc}
\usepackage[TU]{fontenc}
% Lingua utilizada para o documento
%\usepackage[english]{babel}
\usepackage[brazil]{babel}
% Para mais funções de matemática
\usepackage{amsmath}
\usepackage{amssymb}
% Para inserir figuras
\usepackage{graphicx}
% Para colocar links e metadados no PDF
\usepackage{hyperref}
% Para melhorar a distribuição de texto
\usepackage{microtype}
% Para citações
\usepackage[round,authoryear,sort]{natbib}
% Para digitar unidades
\usepackage{siunitx}
% Para incluir URLs com links e quebra de linha adequada
\usepackage{xurl}

% Configurações de formato do documento (chama funções dos pacotes acima)
% %%%%%%%%%%%%%%%%%%%%%%%%%%%%%%%%%%%%%%%%%%%%%%%%%%%%%%%%%%%%%%%%%%%%%%%%%%%%%
% Configura os metadados do PDF
\hypersetup{
    % Links coloridos ao invés da caixa colorida
    colorlinks,
    % Cor dos links
    allcolors=blue,
    % Título que aparece nos metadados
    pdftitle={Mudança na temperatura média de países nos últimos cinco anos},
    % Autor que aparece nos metadados
    pdfauthor={Leonardo Uieda, Yago M. Castro, Arthur S. Macêdo},
    % Permite quebrar a linha no meio de links
    breaklinks=true,
}

% Incluir as variáveis geradas pelo cógido
% %%%%%%%%%%%%%%%%%%%%%%%%%%%%%%%%%%%%%%%%%%%%%%%%%%%%%%%%%%%%%%%%%%%%%%%%%%%%%
\input{variaveis/n_paises.tex}
\input{variaveis/paises.tex}


% Começa o documento em si
% %%%%%%%%%%%%%%%%%%%%%%%%%%%%%%%%%%%%%%%%%%%%%%%%%%%%%%%%%%%%%%%%%%%%%%%%%%%%%
\begin{document}

\title{Mudança na temperatura média de países nos últimos cinco anos}
\author{
    Leonardo Uieda\textsuperscript{1,2},
    Yago M. Castro\textsuperscript{1,2},
    Arthur S. Macêdo\textsuperscript{1,2}
    \\[0.2cm]
    {\small
        \textsuperscript{1}Computer-Oriented Geoscience Lab;
        \textsuperscript{2}Universidade de São Paulo, Brasil
    }
}
\date{\today}

\maketitle

\begin{abstract}
Mudanças climáticas estão afetando o mundo todo.
Dados de temperatura na superfície da Terra são fundamentais para entendermos como este fenômeno impacta diferentes lugares do mundo.
Analisamos séries temperais de temperatura média mensal de \NPaises{} países.
Com essa análise, determinados quais são os países com maior variação de
temperatura nos últimos cinco anos.
\end{abstract}

\section{Introdução}

Sabemos que a temperatura média da Terra tem aumentado durante o Holoceno, com
as mudanças mais recentes sendo consideradas anômalas \citet{Osman2021}.

\section{Metodologia}

\DeclareSIUnit{\ano}{ano}

Seja a taxa variação de temperatura $\alpha$ em \unit{\degreeCelsius\per\ano},
assumimos uma variação linear da temperatura $T$ com o tempo $t$,

\begin{equation}
    T(t) = t \alpha + \beta
    \label{eq:linear}
\end{equation}

\noindent
em que $\beta$ é o coeficiente linear da reta.
Podemos ajustar a equação \ref{eq:linear} aos dados dos últimos cinco anos
e calcular a taxa de variação $\alpha$.

\section{Resultados}

Analisamos os dados de \NPaises{} países: \Paises{}.

% TODO: Inserir figura
%
%
\section{Conclusões}

O mundo está pegando fogo e precisamos parar com a nossa dependência nos
combustíveis fósseis.

\section*{Agradecimentos}

Agradecemos a todas as pessoas que dedicaram seu tempo à criação dos softwares
livres dos quais dependemos para a criação desse artigo.

\bibliographystyle{apalike}
\bibliography{referencias}

\end{document}
